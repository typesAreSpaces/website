\documentclass{article}

\usepackage{symbols}

\input{kldb}
\makeindex
\makeglossaries

\begin{document}

\title{Not not Law of Excluded Middle and Glivenko's translation}
\author{Jose Abel Castellanos Joo\\Department of Computer Science\\University
of New Mexico\\}

\date{\today}
\maketitle
%\tableofcontents

In intuitionistic logic, the intuitionistic negation $\neg A$ is encoded as $A
\rightarrow \bot$. 

\section{Natural Deduction for Classical Logic}

\begin{definition} The following rules definte Classical Logic:

  \begin{prooftree}
    \hypo{\varphi}
    \hypo{\psi}
    \infer2[Intro-$\land$]{\varphi \land \psi}
  \end{prooftree}

  \begin{prooftree}
    \hypo{\varphi \land \psi}
    \infer1[Elim-$\land$1]{\varphi}
  \end{prooftree}

  \begin{prooftree}
    \hypo{\varphi \land \psi}
    \infer1[Elim-$\land$2]{\psi}
  \end{prooftree}

  \begin{prooftree}
    \infer0[a]{\varphi}
    \ellipsis{}{\psi}
    \infer1[Intro-$\rightarrow$a]{\varphi \rightarrow \psi}
  \end{prooftree}

  \begin{prooftree}
    \hypo{\varphi}
    \hypo{\varphi \rightarrow \psi}
    \infer2[Elim-$\rightarrow$]{\psi}
  \end{prooftree}

  \begin{prooftree}
    \hypo{\bot}
    \infer1[$\bot$]{\varphi}
  \end{prooftree}

  \begin{prooftree}
    \infer0[a]{\neg \varphi}
    \ellipsis{}{\bot}
    \infer1[$RAA$]{\varphi}
  \end{prooftree}

  \begin{prooftree}
    \hypo{\varphi}
    \infer1[Intro-$\lor$1]{\varphi \lor \psi}
  \end{prooftree}

  \begin{prooftree}
    \hypo{\psi}
    \infer1[Intro-$\lor$2]{\varphi \lor \psi}
  \end{prooftree}

  \begin{prooftree}
    \hypo{\varphi \lor \psi}
    \infer0[]{\varphi}
    \ellipsis{}{\sigma}
    \infer0[]{\psi}
    \ellipsis{}{\sigma}
    \infer3[Elim-$\lor$]{\sigma}
  \end{prooftree}
  
\end{definition}

\section{Natural Deduction for Intuitionistic Logic}

We just drop the rule $RAA$. The rule $RAA$ stands for the `reductio ad
absurdum' rule. 

\section{Theorems}

\begin{theorem}
  $\vdash_I A \rightarrow \neg \neg A$
\end{theorem}

\begin{proof}

  \begin{prooftree}
    \infer0[1]{A}
    \infer0[2]{A \rightarrow \bot}
    \infer2[Elim-$\rightarrow$]{\bot}
    \infer1[Intro-$\rightarrow$2]{(A \rightarrow \bot) \rightarrow \bot}
    \infer1[Intro-$\rightarrow$1]{A \rightarrow ((A \rightarrow \bot) \rightarrow \bot)}
  \end{prooftree}
  
\end{proof}

\begin{theorem}
  $\vdash_I \neg\neg(A \lor \neg A)$
\end{theorem}

\begin{proof}

  \begin{prooftree}
    \infer0[1]{A}
    \infer1[Left intro $\lor$]{A \lor (A \rightarrow \bot)}
    \infer0[2]{(A \lor (A \rightarrow \bot)) \rightarrow \bot}
    \infer2[MP]{\bot}
    \infer1[Intro-1]{A \rightarrow \bot}
    \infer1[Right intro $\lor$]{A \lor (A \rightarrow \bot)}
    \infer0[2]{(A \lor (A \rightarrow \bot)) \rightarrow \bot}
    \infer2[MP]{\bot}
    \infer1[Intro-2]{((A \lor (A \rightarrow \bot)) \rightarrow \bot)
      \rightarrow \bot} 
  \end{prooftree} 

\end{proof}

\begin{theorem}
  If $\vdash_I \neg \neg A$ and $\vdash_I \neg \neg (A \rightarrow B)$,then
  $\vdash_I \neg \neg B$ 
\end{theorem}

\begin{proof}

  \begin{prooftree}
    \infer0[1]{A}
    \infer0[2]{A \rightarrow B}
    \infer2[MP]{B}
    \infer0[3]{B \rightarrow \bot}
    \infer2[MP]{\bot}
    \infer1[Intro-1]{A \rightarrow \bot}
    \infer0[Hyp]{(A \rightarrow \bot) \rightarrow \bot}
    \infer2[MP]{\bot}
    \infer1[2]{(A \rightarrow B) \rightarrow \bot}
    \infer0[Hyp]{((A \rightarrow B) \rightarrow \bot) \rightarrow \bot}
    \infer2[MP]{\bot}
    \infer1[Intro-3]{(B \rightarrow \bot) \rightarrow \bot}
  \end{prooftree}

\end{proof}

% \bibliographystyle{plain}
% \bibliography{./../../../references}
% \printglossaries
% \printindex
\end{document}

